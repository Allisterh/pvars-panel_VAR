
\section{Summary} \label{sec:Summary}
This article has presented a set of VAR methods for panel data (Section~\ref{sec:Review}), described their implementation in \pkg{pvars} (Section~\ref{sec:Implementation}), and illustrated their application (Section~\ref{sec:Illustration}). The \proglang{R}-package comprises \textit{panel cointegration rank tests} as well as \textit{estimators of VAR models for heterogeneous panels} and \textit{panel methods for structural identification} of the reduced-form VAR models. In this context, \pkg{pvars} addresses typical properties of financial and macroeconomic panel data, in particular cross-sectional dependence and structural breaks in the deterministic term. Finally, \pkg{pvars} supplements functions for model specification and dynamic analysis which are not provided by other packages of the \pkg{vars}-ecosystem, namely various \textit{criteria for the number of common factors}, \textit{persistence profiles}, \textit{mean-group IRF}, and \textit{moving-block bootstrap procedures} for panel SVAR models.


%%%% OUTLOOK %%%
Future research may extend bootstrapped tests for the individual cointegration rank to  panel tests. \citet{Swensen2006,Swensen2009} and \citet{CavaliereEtAl2012,CavaliereEtAl2014} construct bootstrapped tests for the individual Johansen procedure. \citet{Trenkler2009} and \cite{CavaliereEtAl2013} compare bootstrapped SL-procedures in view of deterministic components. In particular, the bootstrap-after-bootstrap procedure by \citet{CavaliereEtAl2015} exhibits improved small-$T$ sample performance and robustness against conditional heteroskedasticity and serial correlation in comparison to the asymptotic test procedure. An extension based on panel blocks could respect additional cross-sectional dependence and would be easily accommodated into \pkg{pvars}. Yet, these forms of residual structure can often be attributed to common or extraordinary shocks and thus treated by common factors or deterministic dummies \citep[Ch.~6.7]{Juselius2007}. Their rigorous implementation throughout the different VAR applications is readily available in \pkg{pvars}.


